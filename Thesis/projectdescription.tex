The yearly held international Carolo-Cup is a student competition, where student teams let their remote-controlled (RC) model cars compete against each other under realistic but, compared to a real driving scenario, abstracted and simplified conditions. Hereby, a given course has to be driven as fast as possible, whereat additional tasks have to be mastered, as avoiding static and dynamic obstacles, parking and obeying right of way, speed limits and other rules, all fully automated. Hence the real challenge lies in the development and implementation of innovative algorithms for vehicle control and environment perception, which is done by the teams during the year.

However, all teams are, due to the miniaturization of the model cars, faced with the problem of limited processing power and energy supply. Within the scope of this work a method, able to detect and classify static and dynamic obstacles, ground points, ramps and background points in a point cloud acquired by a depth camera, has to be developed and implemented. Moreover, the extent and orientation of dynamic objects has to be estimated by the method, which besides, has to attach great importance to the given constraints of the model car. Finally the method has to be evaluated regardings its detection rate, classification rate and the precision of the detections.